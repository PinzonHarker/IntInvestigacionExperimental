\section{Recuento de probabilidad y estadística}

\subsection{Espacio muestral y eventos}
El espacio muestral $\Omega$ es el conjunto de todos los posibles resultados de un experimento aleatorio. Un evento es un subconjunto de $\Omega$.

\subsection{Probabilidad}
Una función de probabilidad $\prob$ asigna a cada evento $A \subseteq \Omega$ un número entre 0 y 1 tal que:
\begin{itemize}
    \item $\prob(\Omega) = 1$
    \item Si $A_1, A_2, \dots$ son disjuntos, entonces $\prob\left(\bigcup_i A_i\right) = \sum_i \prob(A_i)$
\end{itemize}

\subsection{Ejemplo: Dados}
Al lanzar un dado justo de seis caras, $\Omega = \{1, 2, 3, 4, 5, 6\}$ y cada evento tiene probabilidad $\frac{1}{6}$.

\subsection{Variable aleatoria}
Una variable aleatoria es una función $X: \Omega \rightarrow \mathbb{R}$. Por ejemplo, si $X$ es el valor del dado, entonces $\E[X] = \sum_{i=1}^6 i \cdot \frac{1}{6} = 3.5$

\subsection{Distribuciones importantes}
- Distribución binomial
- Distribución de Poisson
- Distribución normal

\subsection{Estadística}
Estudia cómo obtener conclusiones sobre una población a partir de una muestra.

\subsection{Ejemplo de estadística}
Si lanzamos 100 veces un dado, el promedio observado debería ser cercano a 3.5, pero hay fluctuaciones estadísticas.

